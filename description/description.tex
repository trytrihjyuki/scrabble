\documentclass[11pt]{article}
\usepackage[paper=a4paper,left=24mm,right=24mm,top=20mm,bottom=20mm]{geometry}

\usepackage[polish]{babel}
\usepackage[T1]{fontenc}
\usepackage{polski}
\usepackage[utf8]{inputenc}
\usepackage{amsmath}
\usepackage{graphicx}
\usepackage{wrapfig}
\usepackage{paralist}
\usepackage{xcolor}
\usepackage{mathtools}
\usepackage{hyperref}

\hypersetup{
    colorlinks=true,
    linkcolor=teal,
    filecolor=magenta,      
    urlcolor=blue,
}

\pagestyle{empty}                                   % No pagenumbers/headers/footers

%%% Custom sectioning (sectsty package)
\usepackage{sectsty}
\sectionfont{%                                      % Change font of \section command
    \usefont{T1}{phv}{m}{n}%                       
    \sectionrule{0pt}{0pt}{-10pt}{1pt}
}
\subsectionfont{%                                   % Change font of \subsection command
    \usefont{T1}{phv}{m}{n}%                       
    %\sectionrule{0pt}{0pt}{-10pt}{1pt}
}

%%% Macros
\newcommand{\sepspace}{\vspace*{1.3em}}               % Vertical space macro
\newcommand{\NewPart}[1]{\section*{\Large{#1}}}
\newcommand{\NewSubPart}[1]{\subsection*{#1}}

\newcommand{\Tittle}[1]{ 
    \noindent
    \Huge\usefont{T1}{phv}{m}{n} #1  \hfill        % Name
    \par \normalsize \normalfont
}

\newcommand{\SimpleEntry}[1]{
    \noindent\hangafter=0          % Indentation
    #1 \par                                         % Entry 
}

\newcommand{\Description}[1]{
    \noindent\textbf{#1}
    \par
    \normalsize
}

\begin{document}
\begin{wrapfigure}{r}{0.3\textwidth}
    \vspace*{-0.5cm}
    \vspace{-5cm}
\end{wrapfigure}

\Tittle{Scrabble}

\sepspace
\sepspace
\sepspace
\sepspace

\Description{
Program do grania w popularną grę \href{https://pl.wikipedia.org/wiki/Scrabble}{Scrabble}. Gra będzie spełniała wszystkie warunki zapisane w oficjalnych zasadach Scrabble. Do gry jest dołączony jest słownik sjp (baza ok. 4 mln. słów, włącznie z ich odmianami).
Możliwa jest rozgrywka dla 4 graczy. Graczem może być człowiek lub program (docelowo z różnymi poziomami zaawansowania).
}

\sepspace

\Description{
W ustawieniach gry będzie można:
\begin{itemize}
\item zmieniać maksymalny czas na zrobienie ruchu
\item ustawiać nazwy graczy
\item zmieniać wersję językową (polski/angielski)
\end{itemize}
}

\sepspace

\Description{
Uruchamiając program widoczna będzie plansza, przycisk uruchamiający grę, przycisk włączający ustawienia, przycisk uruchamiający program do podpowiedzi oraz przycisk do słownika.
Podczas rozgrywki będzie widoczna plansza i czas do końca ruchu. Wyświetlana też będzie nazwa gracza, którego jest tura oraz jego dostępne litery, poniżej planszy.\newline W ustawieniach widoczne będą okienka do zmiany odpowiednich parametrów gry.
}

\sepspace

\Description{
Kolejną funkcją jest możliwa podpowiedź. Program analizuje wszystkie możliwe kombinacje z liter na planszy i danego gracza, szuka najlepszego możliwego ruchu. Na bazie podpowiedzi swoje ruchy będzie wykonywał program jako gracz.
}

\sepspace

\Description{
Wykonywanie ruchu będzie polegało na wciskaniu myszką w dostępne literki (w odpowiedniej kolejności) a później wybór wyrazu początku na planszy (wybór poziomej/pionowej orientacji). W razie pomyłki przyciskiem wróć wraca się do początku ruchu. Program będzie sprawdzał poprawność wykonywanego ruchu. \newline Zgodnie z zasadami będzie w ramach ruchu będzie można wymienić literki.
}

\sepspace

\Description{
Za obsługę interfejsu graficznego aplikacji odpowiedzialna będzie biblioteka \href{https://www.sfml-dev.org/index.php}{SFML}.
}

\sepspace
\Description{
Oprócz gry program będzie udostępniał możliwość podpowiedzi poza rozgrywką oraz słownik. To znaczy po wpisaniu liter program będzie podawał najwyżej punktowane słowo możliwe do ułożenia z tych liter. Słownik po prostu będzie sprawdzał czy wpisane słowo jest poprawne i można go użyć w grze. Te funkcję mogą służyć do gry w Scrabble poza programem, np. w grze ze znajomymi przy stole.
}


\end{document}